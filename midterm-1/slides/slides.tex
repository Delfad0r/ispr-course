\documentclass[10pt]{beamer}
\usepackage{mystyle}

\midterm{1}
\assignment{3}
\date{March 30, 2021}

\title{Pitch Detector}

\DeclareMathOperator{\Auto}{Auto}
\DeclareMathOperator{\reverse}{reverse}


\begin{document}

\def\yy{\mathbf{y}}

\frame{\titlepage}
\begin{frame}[fragile]{Autocorrelogram}
\begin{itemize}
\item The autocorrelogram $\Auto_{\yy}$ measures the correlation of a signal $\yy$ with itself at different time lags:
\[
\Auto_{\yy}[\tau]=\frac{1}{\norm{\yy}^2}\sum_{t=0}^{N-\tau-1}\yy[t]\cdot\yy[t+\tau].
\]
\item It can be computed by convolution, but for a small set of time lags (\pyth{window}), the naive implementation is more efficient.
\end{itemize}
\begin{python}
def autocorrelogram(y, window):
    b, e = window
    a = np.array([np.dot(y[0 : y.size - tau],y[tau :])
        for tau in range(b, e)])
    return a / a[0]
\end{python}
\end{frame}
\begin{frame}[fragile]{Finding the Pitch}
\begin{itemize}
\item Peaks in the autocorrelogram correspond to periods of the signal $\yy$.
\item The minimal period $\tau_0$ of $\yy$ is the smallest maximum point of the autocorrelogram \textbf{after 0}.
\end{itemize}
\begin{python}
def find_pitch(y, sr, lowest_freq = 80.):
    cor = autocorrelogram(y, [0, int(sr / lowest_freq)])
    peaks = []
    for a in np.split(np.arange(cor.size),
		np.nonzero(cor < 0)[0].tolist())[1 :]:
	        u = [i for i in a if cor[i] > .01]
	        if u:
	            peaks.append(max(u, key = lambda i: cor[i]))
    highest_peak = max(cor[p] for p in peaks)
    f = np.array([p for p in peaks
			if cor[p] > .95 * highest_peak])
    d = np.average(f / np.arange(1, f.size + 1), 0, cor[f])
    return sr / d
\end{python}
\end{frame}
\begin{frame}[fragile]{Results}
\def\importautocorrelogram#1{\raisebox{-.4\height}{\resizebox{!}{1.3cm}{\input{pictures/#1.pgf}}}}
\begin{tabularx}{\textwidth}{@{}ll>{\centering\arraybackslash}Xrr@{}}\toprule
Instrument &  Note & Autocorrelogram & Pitch & Error \\\midrule\addlinespace
Clarinet & C6 & \importautocorrelogram{clarinet_c6} & \SI{1049}{\hertz} & \SI{.2}{\percent}\\\addlinespace
Oboe & C6 & \importautocorrelogram{oboe_c6}& \SI{1046}{Hz} & \SI{.01}{\percent}\\\addlinespace
\makecell[lt]{Keyboard\\\scriptsize(homemade)} & G3 & \importautocorrelogram{keyboard_g3} & \SI{196.7}{\hertz} & \SI{.3}{\percent}\\\addlinespace\bottomrule
\end{tabularx}
\end{frame}
\begin{frame}{Real-time Pitch Detection}
content...
\end{frame}
\end{document}